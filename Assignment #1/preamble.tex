\usepackage{setspace}
\usepackage[rgb]{xcolor}
\usepackage{verbatim}
\usepackage{amsgen,amsmath,amstext,amsbsy,amsopn,tikz,amssymb,tkz-linknodes}
\usetikzlibrary{arrows, calc, babel, shapes}
\usepackage{fancyhdr}
\usepackage[colorlinks=true, urlcolor=blue,  linkcolor=blue, citecolor=blue]{hyperref}
\usepackage[colorinlistoftodos]{todonotes}
\usepackage[fulladjust]{marginnote}
\usepackage{rotating}
\usepackage[utf8]{inputenc}
\usepackage[spanish]{babel}
\usepackage{paracol}
\usepackage{tkz-fct}
\usepackage{fancyhdr}
\usepackage{xstring}
\usepackage{geometry}
\geometry{
    left = .5in,
    right = 1.5in,
    top = 1in,
    head = 1cm,
    foot = .5in,
    bindingoffset=0in,
}
\pagestyle{fancy}
\lhead{\textbf{Penn State University}}
\chead{\includegraphics[scale=.03]{penn}}
\rhead{\textbf{Dept. of IST}}
\graphicspath{{images/}}
\hypersetup{%
pdfauthor={Amirreza Bagherzadeh},%
pdftitle={Homework},%
pdfkeywords={Tikz,latex,bootstrap,uncertaintes},%
pdfcreator={PDFLaTeX},%
pdfproducer={PDFLaTeX},%
}
%\usetikzlibrary{shadows}

\usepackage{booktabs}
\newcommand{\ra}[1]{\renewcommand{\arraystretch}{#1}}

      \newtheorem{thm}{Teorema}[section]
      \newtheorem{prop}[thm]{Proposición}
      \newtheorem{lem}[thm]{Lema}
      \newtheorem{cor}[thm]{Corolario}
      \newtheorem{defn}[thm]{Definición}
      \newtheorem{rem}[thm]{Nota}
      \numberwithin{equation}{section}

\newcommand{\homework}[6]{
   \pagestyle{myheadings}
   \thispagestyle{fancy}
   \newpage
   \setcounter{page}{1}
   \noindent
   \begin{center}
   \framebox{
      \vbox{\vspace{2mm}
    \hbox to 1\textwidth { {\bf{Deep learning} \hfill  PSUID:946204639} }
       \vspace{6mm}
       \hbox to 1\textwidth { {\Large \hfill #1 (#2)  \hfill} }
       \vspace{6mm}
       \hbox to 1\textwidth { {\it Professor: #3 \hfill Student: #5} }
      \vspace{2mm}}
   }
   \end{center}
   \markboth{#5 -- #1}{#5 -- #1}
   \vspace*{4mm}
}

\newcommand{\bbF}{\mathbb{F}}
\newcommand{\bbX}{\mathbb{X}}
\newcommand{\bI}{\mathbf{I}}
\newcommand{\bX}{\mathbf{X}}
\newcommand{\bY}{\mathbf{Y}}
\newcommand{\bepsilon}{\boldsymbol{\epsilon}}
\newcommand{\balpha}{\boldsymbol{\alpha}}
\newcommand{\bbeta}{\boldsymbol{\beta}}
\newcommand{\0}{\mathbf{0}}

\newcommand{\problem}[1]{
    {\begin{tikzpicture}[outline/.style={draw=red!75!gray,thick,fill=red!80!white}]
\node [outline=red] at (0,1) {\bf Problem #1};
\end{tikzpicture}}
}

\newcommand{\solution}[1]{
    {\begin{tikzpicture}[outline/.style={draw=green!75!gray,thick,fill=green!80!white}]
\node [outline=red] at (0,1) {\bf Solution};
\end{tikzpicture}}
}


\newcommand{\complexity}[1]{
    {\begin{tikzpicture}[outline/.style={draw=blue!75!gray,thick,fill=blue!80!white}]
\node [outline=blue] at (0,1) {\bf Complexity};
\end{tikzpicture}}
}

\definecolor{myBlue}{HTML}{027FDF}
\definecolor{negative}{HTML}{181818}
\definecolor{positive}{HTML}{AA3939}


\newcommand{\reaLine}[2]{
    \node (li) at (#1 - 1.5, 0) {};
    \node (ls) at (#2 + 1.5, 0) {$\mathbb{R}$};
	\draw [>=stealth, <->] (li) -- (ls);
}
\newcommand{\interval}[7]{
    \IfEqCase {#6}{
        {nonInf}{
        \node [circle, draw, #4, line width = 1.5pt, color = #7, inner sep = 0pt, minimum size = 5pt] (ci) at (#1, #3) {};
        \node [circle, draw, #5, line width = 1.5pt, color = #7, inner sep = 0pt, minimum size = 5pt] (cs) at (#2, #3) {};
        \draw [line width = 1.5pt, color = #7] (ci) -- (cs);
        \draw (ci) -- (#1, -.2);
        \node (tag) at (#1, -.4) {#1};
        \draw (cs) -- (#2, -.2);
        \node (tag) at (#2, -.4) {#2};
        }
        {inf}{
        	\IfEqCase {#4}{
            	{<-}{
			        \node [circle, draw, #5, line width = 1.5pt, color = #7, inner sep = 0pt, minimum size = 5pt] (cs) at (#2, #3) {};
                    \draw [>=stealth, #4, line width = 1.5pt, color = #7] (#1 - 1.5, #3) -- (cs);
                    \draw (cs) -- (#2, -.2);
                    \node (tag) at (#2, -.4) {#2};
                }
                {->}{
                	\node [circle, draw, #5, line width = 1.5pt, color = #7, inner sep = 0pt, minimum size = 5pt] (ci) at (#1, #3) {};
                    \draw [>=stealth, ->, line width = 1.5pt, color = #7] (ci) -- (#2 + 1.5, #3);
                    \draw (ci) -- (#1, -.2);
                    \node (tag) at (#1, -.4) {#1};
                }
            }
        }
        }[\PackageError{tree}{Undefined option to intervals: #6}{}]
}
%----------------------------------------------------